\documentclass[a4paper,10pt]{article}

\usepackage{fancyhdr}
\usepackage{euler}
\usepackage{capt-of}
\usepackage{amsmath, amscd, amssymb, amsthm}
%\usepackage[frame,cmtip,arrow,matrix,line,graph,curve]{xy}
\usepackage{graphpap, color}
\usepackage[mathscr]{eucal}
\usepackage{paralist}
\usepackage{indentfirst}
\usepackage{hyperref}
%\usepackage{refcheck}
\usepackage{xcolor}


\linespread{1.0}
\setlength{\parindent}{0pt}
\setlength{\parskip}{1.3ex plus 0.2ex minus 0.2ex}
\addtolength{\textwidth}{5cm}
\addtolength{\hoffset}{-2.5cm}
\setlength{\topmargin}{-.7cm}
\setlength{\headsep}{15pt}
\setlength{\textheight}{9.5in}

\pagestyle{fancy} \rhead{Abstracts} \lhead{An Analyst, a Geometer and a Probabilist Walk Into a Bar}
\renewcommand\headrulewidth{0.5pt}
\renewcommand\footrulewidth{0.4pt}

\newcommand{\R}{\mathbb{R}}

\begin{document}

\begin{center}
{\Large\textbf{An Analyst, a Geometer and a Probabilist Walk Into a Bar\\~\\\underline{Abstracts}}}\\
\end{center}


\bigskip


\textbf{Alexander Balinsky (Cardiff)}

\colorbox{gray!20}{\textit{The Analysis and Geometry of Hardy type Inequality and Applications}}

In this talk we present advances that have been made over recent decades in areas of research featuring Hardy's inequality and related topics. The inequality and its extensions and refinements are not only of intrinsic interest but are indispensable tools in many areas of mathematics and mathematical physics. We will also discuss Hardy-type inequalities involving magnetic fields, zero modes and Hardy, Sobolev and Cwikel-Lieb-Rosenblum inequalities for Pauli and Dirac operators.\\




\textbf{Sylvie Benzoni-Gavage (Lyon \& Institut Henri Poincar\'e)}

\colorbox{gray!20}{\textit{***}}

***\\



\textbf{Matteo Capoferri (University College London)}

\colorbox{gray!20}{\textit{The wave propagator between analysis and geometry}}

In my talk I will discuss how it is possible, in the spirit of some classical results due to Laptev, Safarov and Vassiliev, to write the propagator of a class of hyperbolic operators on manifolds as one single oscillatory integral with complex-valued phase function, global both in space and in time. In particular, a more refined, geometric version of the method will be presented, in the Riemannian setting: the adoption of a distinguished complex-valued phase function, naturally dictated by the geometric framework, will allow us to visualise the process of circumventing topological obstructions. The microlocal method is explicit and constructive; the calculation of the subprincipal symbol of the propagator enables us to recover asymptotic spectral properties of the operators at hand. I will discuss explicit formulae and recent results for the wave operator.

This is joint work with D. Vassiliev and M. Levitin. \\

\textbf{Michele Coti Zelati (Imperial College London)}

\colorbox{gray!20}{\textit{Diffusion and mixing in incompressible flows}}

We study diffusion and mixing in different fluid dynamics models, mainly related to incompressible flows. In this setting, 
mixing is a purely advective effect which causes a transfer of energy to high frequency. In turn, mixing acts to enhance 
the dissipative forces, giving rise to what we refer to as enhanced dissipation: this can be understood by the identification 
of a time-scale faster than the purely diffusive one. We will give a general quantitative criterion that links mixing rates 
(in terms of decay of negative Sobolev norms) to enhanced dissipation time-scales. Applications include passive scalar 
evolution in both planar and radial settings, fractional diffusion, linearized two-dimensional Navier-Stokes equations, and 
even simple examples in kinetic theory.\\


\textbf{Hong Duong (Imperial College London)}

\colorbox{gray!20}{\textit{Quantification of coarse-graining error in Langevin and overdamped Langevin dynamics}}

Coarse-graining or dimension reduction is the procedure of approximating a large and complex system by a simpler and lower dimensional one, where the variables in the reduced model are called coarse grained or collective variables. Such a reduction is necessary from a computational point of view since an all-atom molecular simulation of the complex system is often unable to access information about relevant temporal and/or spatial scales. Further this is also relevant from a modelling point of view as the quantities of interest are often described by a smaller class of features. For these reasons coarse graining has gained importance in various fields and especially in molecular dynamics.


In this work, we will study and quantify the coarse-graining error between the coarse-grained dynamics and an effective dynamics. The effective dynamics is a Markov process on the coarse-grained state space obtained by a closure procedure from the coarse-grained coefficients. We obtain error estimates both in relative entropy and Wasserstein distance, for both Langevin and overdamped Langevin dynamics. The approach allows for vectorial coarse-graining maps. Hereby, the quality of the chosen coarse-graining is measured by certain functional inequalities encoding the scale separation of the Gibbs measure. The method is based on error estimates between solutions of (kinetic) Fokker-Planck equations in terms of large-deviation rate functionals.

 

This is joint work with A. Lamacz, M. Peletier, A. Schlichting, and U. Sharma.\\



\textbf{Amit Einav (TU Vienna)}

\colorbox{gray!20}{\textit{On weak Poincar\'e inequalities and densities of states}}

Poincar\'e inequality, which is probably best known for its applications in PDEs and calculus of variation, is one of the simplest examples to an inequality that lies in the crossroads of Analysis, Probability and Semigroup/Spectral theory. It can be understood as the functional inequality that arises from attempting to understand convergence of the so-called heat flow to its equilibrium state. This approach can be generalised to the setting of Markov semigroups, with a non-positive generator that posses a spectral gap.
A natural question that one can consider is: What happens if the generator doesn�t have a spectral gap? Can we still deduce a rate of convergence from a functional setting?
In this talk we will discuss a new approach to this question and see how an understanding of the way the spectrum of the generator behaves near the origin, in the form of a density of states estimate, can lead to weak Poincar\'e type inequalities, from which a quantitative estimation of convergence can be obtained.
This talk is based on joint work with Jonathan Ben-Artzi (arXiv:1805.08557).\\




\textbf{Josephine Evans (Cambridge)}

\colorbox{gray!20}{\textit{Hypocoercivity in relative entropy for the linear Boltzmann equation on the torus}}

In his influential memoire `Hypocoercivity' Villani introduced a collection of techniques for showing convergence to equilibrium for degenerate equations. Most theorems he proves and subsequent work are in $L^2$ or $H^1$ spaces. However, he also showed convergence to equilibrium directly in relative entropy. This theorem only applies to diffusion equations. I will talk about how to show a similar result for the linear relaxation Boltzmann equation and the ways in which this is similar and different to a diffusion equation.\\

\textbf{Max Fathi (Toulouse)}

\colorbox{gray!20}{\textit{Stein kernels, optimal transport and the CLT}}

Stein kernels are a way of measuring distances between probability
measures, defined via integration by parts formulas. I will present a
connection between these kernels and optimal transport. The main
application is a way of deriving rates of convergence in the classical
central limit theorem using regularity estimates for a variant of the
Monge-Amp\`ere PDE. As an application, we obtain new rates of convergence
for the multi-dimensional CLT, with explicit dependence on the dimension.\\

\textbf{Ivan Gentil (Lyon)}

\colorbox{gray!20}{\textit{A generalization of the Schr\"odinger problem via the Otto calculus}}

The Schr\"odinger problem is a minimization problem of the entropy along paths. We propose to include this problem to a general cost via the Otto calculus, hence to give a nice approximation of the Wasserstein distance. \\


\textbf{Arnaud Guillin (Clermont-Ferrand)}

\colorbox{gray!20}{\textit{An elementary approach for uniform in time propagation of chaos}}

Based on a coupling approach, we prove uniform in time propagation of chaos for weakly interacting mean-field particle systems with possibly non-convex confinement and interaction potentials. The approach is based on a combination of reflection and synchronous couplings applied to the individual particles. It provides explicit quantitative bounds that significantly extend previous results for the convex case.

Joint work with A. Durmus, A. Eberle and R. Zimmer.\\




\textbf{Mikaela Iacobelli (Durham)}

\colorbox{gray!20}{\textit{***}}

***\\


\textbf{Hugo Lavenant (Paris-Orsay)}

\colorbox{gray!20}{\textit{Harmonic mappings valued in the Wasserstein space}}

The Wasserstein space, which is the space of probability measures endowed with the so-called (quadratic) Wasserstein distance coming from optimal transport, can formally be seen as a Riemannian manifold of infinite dimension. We propose, through a variational approach, a definition of harmonic mappings defined over a domain of $\R^n$ and valued in the Wasserstein space. We will show how one can build a fairly satisfying theory which captures some key features of harmonicity and present a numerical scheme to compute such harmonic mappings. Other than a better understanding of the Wasserstein space, the motivation of such a study can be found in geometric data analysis.\\


\textbf{Ivan Moyano (Cambridge)}

\colorbox{gray!20}{\textit{***}}

***\\



\textbf{Filippo Santambrogio (Paris-Orsay)}

\colorbox{gray!20}{\textit{Moment measures, variants, applications and variational approaches}}

We say that $\mu$ is the moment measure of a convex function $u$ if $\mu$ is the image of the $\log$-concave density $\rho = \exp(-u)$ through the gradient of $u$. In a recent paper, D. Cordero-Erausquin and B. Klartag (JFA, 2015) have characterized the set of measures $\mu$ which are moment measures of a function  which is at the same time convex and ``essentially continuous'' (a condition which deals with the points where $u$ tends to infinity, if any), proving existence and uniqueness results for $u$ by a variational method. It is also possible, and this will be the main object of the presentation, to obtain this by an alternative approach where one minimizes a functional in $\rho$, which makes its entropy and some transport costs appear.
After presenting these questions I will move to some variants, where the exponential is replaced by other decreasing functions (in particular, negative powers), and I will mention their links with problems in geometry (finding affine hemispheres, for instance, according to a result by Klartag), and, if possible, few words on the numerical methods that can be used to approach the solutions (this is a work in progress with Q. M\'erigot and B. Klartag).\\




\textbf{Benjamin Schlein (Zurich)}

\colorbox{gray!20}{\textit{Excitation spectrum of Bose-Einstein condensates}}

We consider systems of $N$ bosons confined in a box with volume one and interacting through a potential with
scattering length of the order $1/N$ (Gross-Pitaevskii regime). For non-negative and sufficiently weak interactions, we determine the
low-energy spectrum, i.e. the ground state energy and low-lying excitations, up to errors that vanishes in the limit of large $N$,
confirming the validity of Bogoliubov's theory.\\



\textbf{Alfonso Sorrentino (Rome-``Tor Vergata'')}

\colorbox{gray!20}{\textit{On the Birkhoff conjecture for convex billiards}}

A mathematical billiard is a system describing the inertial motion of a point mass inside a domain, with elastic reflections at the boundary. This simple model has been first proposed by G.D. Birkhoff as a mathematical playground where ``it the formal side, usually so formidable in dynamics, almost completely disappears and only the interesting qualitative questions need to be considered''.
Since then billiards have captured much attention in many different contexts, becoming a very popular subject of investigation. Despite their apparently simple (local) dynamics, their qualitative dynamical properties are extremely non-local. This global influence on the dynamics translates into several intriguing rigidity phenomena, which are at the basis of several unanswered questions and conjectures.
In this talk I shall focus on some of these questions. In particular, I shall describe some recent results related to the classification of integrable billiards (also known as Birkhoff conjecture).
This talk is based on works in collaboration with V. Kaloshin and with G. Huang and V. Kaloshin.\\


\textbf{Dmitri Vassiliev (University College London)}

\colorbox{gray!20}{\textit{Lorentzian elasticity}}

In this talk we develop a new mathematical model of elasticity in the Lorentzian setting. Working on a Lorentzian 4-manifold, we consider a diffeomorphism which is the unknown quantity of our mathematical model. We write down a functional of nonlinear elasticity and vary it subject to the volume preservation constraint. The analysis of our nonlinear field equations produces three main results. Firstly, we show that for Ricci-flat manifolds the linearised field equations are Maxwell's equations in the Lorenz gauge with exact current. Secondly, for Minkowski space we construct explicit massless solutions; these come in two distinct types, right-handed and left-handed. Thirdly, for Minkowski space we construct explicit massive solutions; these contain a positive parameter which has the geometric meaning of quantum mechanical mass and a real parameter which may be interpreted as electric charge. In constructing our solutions we resort to group-theoretic ideas: we identify special 4-dimensional subgroups of the Poincar\'e group and seek diffeomorphisms compatible with their action, in a suitable sense.

This is joint work with Matteo Capoferri, preprint arXiv:1805.01303. \\




\textbf{Simon Zugmeyer (Lyon)}

\colorbox{gray!20}{\textit{***}}

***\\





\end{document}
